% !TEX program=xelatex

% nobabel means you'll need to provide your own month names, week day headings,
% etc
\documentclass[17pt,nobabel,sundayweek,giant,rgb]{cdcalendar}

%% and here we do some settings for a calendar in Chinese
%% (*not* the lunar calendar! Just localising the Gregorian calendar into
%% Chinese)
\usepackage{zh-mod}
\newcommand{\weibo}[1]{{\xeCJKsetup{CJKecglue={}}\faWeibo{}\,@#1}}

\usepackage[rm]{roboto}

\usepackage{csvsimple,xstring}

\robustify{\\}
\newcommand{\annotdate}[2]{%
  \ifboolexpr{ test{\IfBeginWith{#2}{初}}
            or test{\IfBeginWith{#2}{十}}
            or test{\IfEndWith{#2}{十}}
            or test{\IfBeginWith{#2}{廿}}
  }{\def\extra{}}%
   {\def\extra{red!70!black}}%
  \node[font=\tiny\sffamily\CJKspace,\extra,
  anchor=north,outer sep=0.55ex,align=flush center]
  at (cdCal-\thepage-#1) {#2};%
}

\newcommand{\importlunar}[1]{%
\begin{scope}[on background layer]%
  %% lunar-2020.csv contains mappings between the Chinese lunar calendar and
  %% 2020 Gregorian calendar, with columns EVENT and DATE.
  %% Sourced from https://github.com/infinet/lunar-calendar/blob/master/chinese_lunar_prev_year_next_year.ics; converted to Simplified Chinese;
  %% converted to .csv with https://www.indigoblue.eu/ics2csv/
  \csvreader[filter test=\IfBeginWith{\LNDATE}{#1}]
  {lunar-2020.csv}{EVENT=\LNEVENT,DATE=\LNDATE}
  {\annotdate{\LNDATE}{\LNEVENT}}
\end{scope}%
}

%%%%%%
% Some settings for the monthly calendars, to accomodate extra annotations below each date
%%%%%%
\dayHeadingStyle{font=\sffamily\color{gray!90}}
\sundayColor{red}
\monthTitleStyle{font={\fontsize{42pt}{44pt}\bfseries\fangsong},text=胭脂}
\eventStyle{\scriptsize\heiti\def\CJKecglue{}}
\renewcommand\printeventname[1]{{\sffamily #1}}
\tikzset{every calendar/.append style={day yshift=2.5\ccwd,day xshift=2.75\ccwd},every day/.append style={font=\bfseries}}
\patchcmd{\monthCalendar}{0pt,\paperheight}{0pt,\dimexpr\paperheight-0.5cm\relax}{}{}

%% 月份上下加的小点缀,嫌太花了的可以把这些删掉。
\usepackage{scrlfile}
\ReplacePackage{xpatch}{regexpatch}
\usepackage{pgfornament-han}
\usepackage{cncolours}
\patchcmd{\monthCalendar}
  {\pgfcalendarmonthname}
  {\rotatebox{90}{\pgfornamenthan[width=.8\ccwd,color=金色]{43}}\\%
   \pgfcalendarmonthname}
  {}{}
\patchcmd{\monthCalendar}
  {{\rotatebox{-90}{#1}}}
  {{\rotatebox{-90}{#1}}\\[0.75ex]%
    \rotatebox{90}{\pgfornamenthan[width=.8\ccwd,color=缃色]{43}}}
  {}{}

%% Define all event mark styles here
\tikzset{holiday/.style={rectangle,fill=orange!60,inner xsep=1.375em,inner ysep=0.5em}}
\tikzset{pink icon/.style={text=Pink,font=\large}}
\tikzset{blue icon/.style={text=SkyBlue!60,font=\large}}


\usepackage{fontawesome}
\usepackage{graphicx}
\usepackage{wallpaper}
\graphicspath{{img/}}

\setlength{\CalPageMargin}{1cm}
% \setlength{\EventLineWidth}{6in}
\geometry{a4paper}

\begin{document}

%\TileWallPaper{.5\paperwidth}{.5\paperheight}{ricepaper_v3}
\coverImage[\vspace*{-7.25cm}~\weibo{溪畔鹤迹} \url{http://1t.click/bxy3}]{taotie}

\coverTitle[
font=\fontsize{48pt}{50pt}\sffamily\fangsong\bfseries,
text width=\linewidth,align=flush right,枣红]
{2020年历(附农历节气)}

\makeCover
\clearpage

%% Uncomment the next line if you want to clear the background
%\ClearWallPaper

\illustration[西汉,错金银青铜锯齿形器。图片拍摄:一只饭包。\weibo{张博力}\url{http://t.cn/EUm09h0}]{\textwidth}{IMG_4645}
% 春节:1月24日至30日放假调休,共7天。1月19日(星期日)、2月1日(星期六)上班。
\begin{monthCalendar}{2020}{01}
  \event[mark style=holiday]{2020-01-01}{}{元旦}
  \event{2020-01-08}{2020-01-10}{ACME Conference}
  \event[mark style=blue icon,marker=\faBriefcase]{2020-01-19}{}{上班日}
  \event[mark style=pink icon,marker=\faBirthdayCake]{2020-01-23}{}{朋友生日}
  \event[mark style=holiday]{2020-01-24}{2020-01-30}{春节放假调休}
  \importlunar{2020-01}
\end{monthCalendar}

\clearpage

\illustration[商 有领玉璧 1986年广汉三星堆遗址K2出土 四川广汉三星堆博物馆藏 \weibo{川后} \url{http://t.cn/EUmH7He}]{\textwidth}{youlingyupi}

\begin{monthCalendar}{2020}{02}
  \event[mark style=blue icon,marker=\faBriefcase]{2020-02-01}{}{上班日}
  \importlunar{2020-02}
\end{monthCalendar}



\clearpage

\illustration[Immortals Riding Dragons: Section of a Tomb Pediment.
Han dynasty (206 B.C.--A.D. 220), 1st century B.C./A.D. Probably from Henan province, China. Art Initiative Chicago. \url{http://t.cn/EUmmVjV}]{\textwidth}{dragonriders}

\begin{monthCalendar}{2020}{03}
  \importlunar{2020-03}
\end{monthCalendar}

\clearpage

\illustration[明代金饰件,杭州桃源岭出土,浙江省博物馆 藏。\weibo{一梦琥珀}\url{http://1t.click/bxyv}]{\textwidth}{toushi}

\begin{monthCalendar}{2020}{04}
  \event[mark style=holiday]{2020-04-04}{3}{清明节放假调休}
  \event[mark style=blue icon,marker=\faBriefcase]{2020-04-26}{}{上班日}
  \importlunar{2020-04}
\end{monthCalendar}

\clearpage

\illustration[西汉,错金银青铜锯齿形器。图片拍摄:一只饭包。\weibo{张博力}\url{http://t.cn/EUm09h0}]{\textwidth}{IMG_4645}

\begin{monthCalendar}{2020}{05}
  \event[mark style=holiday]{2020-05-01}{5}{劳动节放假调休}
  \event[mark style=blue icon,marker=\faBriefcase]{2020-05-09}{}{上班日}
  \importlunar{2020-05}
\end{monthCalendar}

\clearpage

\illustration[商 有领玉璧 1986年广汉三星堆遗址K2出土 四川广汉三星堆博物馆藏 \weibo{川后} \url{http://t.cn/EUmH7He}]{\textwidth}{youlingyupi}

\begin{monthCalendar}{2020}{06}
  \event[mark style=holiday]{2020-06-25}{3}{端午节放假调休}
  \event[mark style=blue icon,marker=\faBriefcase]{2020-06-28}{}{上班日}
  \importlunar{2020-06}
\end{monthCalendar}

\clearpage

\illustration[Immortals Riding Dragons: Section of a Tomb Pediment.
Han dynasty (206 B.C.--A.D. 220), 1st century B.C./A.D. Probably from Henan province, China. Art Initiative Chicago. \url{http://t.cn/EUmmVjV}]{\textwidth}{dragonriders}

\begin{monthCalendar}{2020}{07}
  \importlunar{2020-07}
\end{monthCalendar}

\clearpage

\illustration[明代金饰件,杭州桃源岭出土,浙江省博物馆 藏。\weibo{一梦琥珀}\url{http://1t.click/bxyv}]{\textwidth}{toushi}

\begin{monthCalendar}{2020}{08}
  \importlunar{2020-08}
\end{monthCalendar}

\clearpage

\illustration[西汉,错金银青铜锯齿形器。图片拍摄:一只饭包。\weibo{张博力}\url{http://t.cn/EUm09h0}]{\textwidth}{IMG_4645}

\begin{monthCalendar}{2020}{09}
  \event[mark style=blue icon,marker=\faBriefcase]{2020-09-27}{}{上班日}
  \importlunar{2020-09}
\end{monthCalendar}

\clearpage

\illustration[商 有领玉璧 1986年广汉三星堆遗址K2出土 四川广汉三星堆博物馆藏 \weibo{川后} \url{http://t.cn/EUmH7He}]{\textwidth}{youlingyupi}

\begin{monthCalendar}{2020}{10}
  \event[mark style=holiday]{2020-10-01}{2020-10-08}{国庆节、中秋节放假调休}
  \event[mark style=blue icon,marker=\faBriefcase]{2020-10-10}{}{上班日}
  \importlunar{2020-10}
\end{monthCalendar}

\clearpage

\illustration[Immortals Riding Dragons: Section of a Tomb Pediment.
Han dynasty (206 B.C.--A.D. 220), 1st century B.C./A.D. Probably from Henan province, China. Art Initiative Chicago. \url{http://t.cn/EUmmVjV}]{\textwidth}{dragonriders}

\begin{monthCalendar}{2020}{11}
  \importlunar{2020-11}
\end{monthCalendar}

\clearpage

\illustration[明代金饰件,杭州桃源岭出土,浙江省博物馆 藏。\weibo{一梦琥珀}\url{http://1t.click/bxyv}]{\textwidth}{toushi}

\begin{monthCalendar}{2020}{12}
  \importlunar{2020-12}
\end{monthCalendar}

\end{document}
