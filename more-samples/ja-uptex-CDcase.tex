% nobabel means you'll need to provide your own month names, week day headings,
% etc. We'll use jsarticle+uplatex in this example.
\RequirePackage{scrlfile}
\ReplaceClass{extarticle}{jsarticle}
\BeforePackage{xcolor}{\RequirePackage[12pt]{extsizes}}
% \RemoveFromHook{top-level}{extarticle}
% \AddToHook{top-level}{jsarticle}
% \AddToHook{class/after/jsarticle}{\RequirePackage[12pt]{extsizes}}
\PassOptionsToClass{uplatex}{jsarticle}
\documentclass[nobabel,sundayweek,dvipdfmx]{cdcalendar}

%% and here we do some settings for a calendar in Japanese
\usepackage{jp-mod-uptex}

\usepackage[rm]{roboto}

\usepackage{graphicx}
\usepackage{fontawesome}
\usepackage{wallpaper}
\usepackage[fleqn]{amsmath}
\usepackage{tikzlings,tikzducks}

\graphicspath{{img/}}

%% Define all event mark styles here
\tikzset{holiday/.style={rectangle,fill=orange!70}}
\tikzset{pink icon/.style={text=Pink,font=\large}}
\tikzset{blue icon/.style={text=SkyBlue,font=\large}}

\begin{document}

%%%%%%
% Cover
%%%%%%

\coverBgColor{RoyalBlue!40!black}
\coverImage[\color{gray!50}ここに実際に印刷されたカレンダーがあります。 小さなカレンダー(9\,cm $\times$ 9.5\,cm)は、フロッピーディスクのジュエルケースに適合します。 大きなもの(11.7\,cm $\times$ 13.65\,cm)はCDのジュエルケースに適合します。]
{actual-crop}

\coverTitle[font=\fontsize{40pt}{42pt}\sffamily\gtfamily,
text=white,text width=\linewidth,align=flush right]
{2024カレンダー}

\makeCover

% Remove this line if you feel the background pattern is too annoying
% \TileWallPaper{.5\paperwidth}{.5\paperheight}{ricepaper_v3}
\TileWallPaper{.5\paperwidth}{.1\paperheight}{lightpaperfibers}

% You may find the gap between illustrations and events too narrow
% Use this length to increase it
\setlength{\illusSkip}{1em}


%%%%%%
% Some settings for the monthly calendars
%%%%%%
\dayHeadingStyle{font=\gtfamily\color{gray!90}}
\sundayColor{red}
\monthTitleStyle{font={\fontsize{48pt}{48pt}\sffamily\gtfamily}, text=RoyalBlue!40!white}
\eventStyle{\color{black}\scriptsize\sffamily\gtfamily\selectfont}

%%%%%%
% January 2020
%%%%%%
\illustration
[Happy TikZ animals! This is an optional description about the illustrations.]
{\linewidth}{tikzlings}

\begin{monthCalendar}{2024}{01}

%%% events must be given AFTER \begin{monthCalendar}
%%% Currently you must give events on the same page
%%% as the monthly calendar.

%% This is an one-day event
\event[mark style=holiday]{2024-01-01}{}{元日}
%% This is a 5-day event
\event[mark style=blue icon,marker=\faBriefcase]{2024-01-30}{5}{ACME会議}
%% you could also write \event{2024-01-06}{2024-02-03}{ACME会議}

\end{monthCalendar}

\clearpage

%%%%%%
% Feb 2020
%%%%%%

% Or you can put any stuff, really, with a caption if you want:
\setlength{\mathindent}{0pt}
\otherstuff[Fourier Transformation, one of the `math equations that changed the world'. \url{http://news.bitofnews.com/13-math-equations-that-changed-the-world/}]
{\linewidth}
{\huge\selectfont
\[%
  \hat{f}(\xi) = \int^{\infty}_{-\infty} f(x) e^{-2\pi ix\xi} \mathop{dx}
\]}

\begin{monthCalendar}{2024}{02}

%% Repeat the event if it spans two months
\event[mark style=blue icon,marker=\faBriefcase]{2024-01-30}{5}{ACME会議}
\event[mark style=pink icon,marker=\faBirthdayCake]{2024-02-07}{}{友人の誕生日}
\event{2024-02-24}{}{締め切り!}

\end{monthCalendar}

\clearpage

%%% I... I was too tired search for more pics so will just show some cute animals
\otherstuff{\linewidth}{\tikz[scale=1.5]{\koala[crown]};}
\begin{monthCalendar}{2024}{03}
\end{monthCalendar}

\clearpage

\otherstuff{\linewidth}{\tikz[scale=1.5]{\duck[graduate,glasses]};}
\begin{monthCalendar}{2024}{04}
\end{monthCalendar}

\clearpage

\otherstuff{\linewidth}{\tikz[scale=1.5]{\mouse[cheese]};}
\begin{monthCalendar}{2024}{05}
\end{monthCalendar}

\clearpage

\otherstuff{\linewidth}{\tikz[scale=1.5]{\duck[crozier,strawhat=red!80!white,bowtie=red]};}
\begin{monthCalendar}{2024}{06}
\end{monthCalendar}

\clearpage

\otherstuff{\linewidth}{\tikz[scale=1.5]{\sloth[icecream]};}
\begin{monthCalendar}{2024}{07}
\end{monthCalendar}

\clearpage

\otherstuff{\linewidth}
  {\tikz[scale=1.5]{\duck[body=pink!50!white,bill=orange,
  unicorn=magenta!60!violet, longhair=magenta!60!violet]};}
\begin{monthCalendar}{2024}{08}
\end{monthCalendar}

\clearpage

\otherstuff{\linewidth}{\tikz[scale=1.5]{\coati[umbrella=blue!60!black,handbag]};}

\begin{monthCalendar}{2024}{09}
\end{monthCalendar}

\clearpage

\otherstuff{\linewidth}
  {\tikz[scale=1.5]{\duck[witch=black!50!gray,
      longhair=red!80!black,
      jacket=black!50!gray,
      magicwand]};}
\begin{monthCalendar}{2024}{10}
\end{monthCalendar}

\clearpage

\otherstuff{\linewidth}{\tikz[scale=1.5]{\duck[snowduck=white!90!gray,eye=white]};}
\begin{monthCalendar}{2024}{11}
\end{monthCalendar}

\clearpage

\otherstuff{\linewidth}{\tikz[scale=1.5]{\marmot[santa,wine]};}
\begin{monthCalendar}{2024}{12}
\end{monthCalendar}

\end{document}
