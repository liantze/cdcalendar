%!TEX program=xelatex

% nobabel means you'll need to provide your own month names, week day headings,
% etc
\documentclass[12pt,nobabel,sundayweek]{cdcalendar}
%% and here we do some settings for a calendar in Chinese
%% (*not* the nongli calendar! Just localising the Gregorian calendar into
%% Chinese)
\usepackage{zh-mod}

%% Use Roboto and Roboto Slab
\usepackage[rm,light,osf,scale=0.94]{roboto}

% \title{Desktop Calendar (fits CD jewel case) with Chinese Localisation}

\usepackage{graphicx}
\usepackage{fontawesome}
\usepackage[fleqn]{amsmath}

\graphicspath{{img/}}

%% Define all event mark styles here
\tikzset{holiday/.style={rectangle,fill=orange!70}}
\tikzset{pink icon/.style={text=Pink,font=\large}}
\tikzset{blue icon/.style={text=SkyBlue,font=\large}}

\usepackage{csvsimple,xstring}

\robustify{\\}
\newcommand{\annotdate}[2]{%
  \ifboolexpr{ test{\IfBeginWith{#2}{初}}
            or test{\IfBeginWith{#2}{十}}
            or test{\IfEndWith{#2}{十}}
            or test{\IfBeginWith{#2}{廿}}
  }{\def\extra{}}%
   {\def\extra{red!70!black}}%
  \node[font=\tiny\sffamily\CJKspace,\extra,
  anchor=north,outer sep=0.55ex,align=flush center]
  at (cdCal-\thepage-#1) {#2};%
}

\newcommand{\importnongli}[1]{%
\begin{scope}[on background layer]%
  %% nongli-2024.csv contains mappings between the Chinese nongli calendar and
  %% 2024 Gregorian calendar, with columns EVENT and DATE.
  %% Created with https://github.com/infinet/nongli-calendar; converted to Simplified Chinese;
  %% converted to .csv with https://www.indigoblue.eu/ics2csv/
  \csvreader[filter test=\IfBeginWith{\LNDATE}{#1}]
  {nongli-2024.csv}{EVENT=\LNEVENT,DATE=\LNDATE}
  {\annotdate{\LNDATE}{\LNEVENT}}
\end{scope}%
}

\begin{document}


%%%%

% You may find the gap between illustrations and events too narrow;
% Use this length to incrase it
\setlength{\illusSkip}{1.5\ccwd}


%%%%%%
% Some settings for the monthly calendars
%%%%%%
\dayHeadingStyle{font=\sffamily\color{gray!90},yshift=-1em}
\sundayColor{red!80!black}
\monthTitleStyle{font={\fontsize{42pt}{42pt}\bfseries\sffamily\fangsong}, text=RoyalBlue!40!white}
\eventStyle{\scriptsize\songti\color{black}}
\renewcommand\printeventname[1]{{\heiti\color{black}#1}}
\tikzset{every calendar/.append style={day yshift=2.5\ccwd,day xshift=2.75\ccwd},every day/.append style={font=\bfseries}}
\patchcmd{\monthCalendar}{0pt,\paperheight}{0pt,\dimexpr\paperheight-0.5em\relax}{}{}




%%%%%%
% January 2024
%%%%%%
\illustration
[Happy TikZ animals! This is an optional description about the illustrations.]
{\linewidth}{tikzlings}

\begin{monthCalendar}{2024}{01}

%%% events must be given AFTER \begin{monthCalendar}
%%% Currently you must give events on the same page
%%% as the monthly calendar.

%% This is an one-day event
\event[mark style=holiday]{2024-01-01}{}{新年元旦}
%% This is a 5-day event
\event[mark style=blue icon,marker=\faBriefcase]{2024-01-30}{5}{ACME大会}
%% you could also write \event{2024-01-06}{2024-02-03}{ACME 大会}

\importnongli{2024-01}
\end{monthCalendar}

\clearpage

%%%%%%
% Feb 2024
%%%%%%

% Or you can put any stuff, really, with a caption if you want:
\setlength{\mathindent}{0pt}
\otherstuff[Fourier Transformation, one of the `math equations that changed the world'. \url{http://news.bitofnews.com/13-math-equations-that-changed-the-world/}]
{\linewidth}
{\huge\selectfont
\[%
  \hat{f}(\xi) = \int^{\infty}_{-\infty} f(x) e^{-2\pi ix\xi} \mathop{dx}
\]}

\begin{monthCalendar}{2024}{02}

%% Repeat the event if it spans two months
\event[mark style=blue icon,marker=\faBriefcase]{2024-01-30}{5}{ACME大会}
\event[mark style=pink icon,marker=\faBirthdayCake]{2024-02-07}{}{朋友生日}
\event{2024-02-24}{}{死线!!}

\importnongli{2024-02}
\end{monthCalendar}

\end{document}
